\numberlist{

	\item \question{How do you create a class in Python?}
	\answer{
	
		To create a class in python, we use the keyword “class” as shown in the example
below:		
		
		\codeblock{
			class InterviewbitEmployee:
\\
			\s def \_\_init\_\_(self, emp\_name): \\
			\s \s self.emp\_name = emp\_name
		}
		
		
		To instantiate or create an object from the class created above, we do the following:
		
		\codeblock{
			emp\_1=InterviewbitEmployee("Mr. Employee")
		}
	
		To access the name attribute, we just call the attribute using the dot operator as shown below:
		
		\codeblock{
			print(emp\_1.emp\_name)
 \\
			\# Prints Mr. Employee
		}
	
		To create methods inside the class, we include the methods under the scope of the
class as shown below:
		
		\codeblock{
			class InterviewbitEmployee:
\\
			\s def \_\_init\_\_(self, emp\_name):
 \\
			\s \s self.emp\_name = emp\_name
 \\
			\s def introduce(self):
 \\
			\s \s print("Hello I am " + self.emp\_name)
		}

	}



	\item \question{How does inheritance work in python? Explain it with an
example.}
	\answer{
	Inheritance gives the power to a class to access all attributes and methods of another
class. It aids in code reusability and helps the developer to maintain applications
without redundant code. The class inheriting from another class is a child class or
also called a derived class. The class from which a child class derives the members are
called parent class or superclass.
	
	Python supports different kinds of inheritance, they are:
	
	\newlist{
		\item \textbf{Single Inheritance}: Child class derives members of one parent class.
	
		\codeblock{ \\
			\# Parent class
 \\
			class ParentClass:
 \\
			\s def par\_func(self):
 \\
			\s \s print("I am parent class function")
 \\
			\\
			\# Child class
 \\
			class ChildClass(ParentClass):
 \\
			\s def child\_func(self):
 \\
			\s \s print("I am child class function")
 \\
			\\
			\# Driver code
 \\
			obj1 = ChildClass()
 \\
			obj1.par\_func()
 \\
			obj1.child\_func()
		}
		
	}
	
	
	
	}



}
